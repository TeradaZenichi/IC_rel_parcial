\section{Introdução}

Os sistemas de distribuição de energia elétrica (SDEE) são planejados como redes malhadas interconectadas.
Entretanto, operam como com uma topologia radial a fim da facilitar a coordenação da proteção e reduzir a corrente de curto circuito dos SDEE.

Para obter uma topologia radial existem chaves em pontos estratégicos do sistema de distribuição de energia elétrica que interconectam seus nós.
Desse modo, a topologia inicial pode ser modificada pela operação das chaves para transferir as demandas entre os diferentes alimentadores e, assim, é possível determinar uma nova topologia com outro ponto de operação, contudo deve continuar sendo uma topologia radial.

O problema de reconfiguração das redes de distribuição (RSD) é um problema de planejamento da operação das chaves alocadas ao longo dos alimentadores e consiste na abertura e/ou fechamento das chaves com o objetivo de melhorar um índice de desempenho.
Dependendo dos tipos, da quantidade e a alocação das chaves, sua comutação pode ser aproveitada por empresas de distribuição com a finalidade de minimizar, por exemplo as perdas de potência ativa no sistema. Entretanto existe diversas restrições operacionais que os operadores da rede devem respeitar na hora de reconfigurar a topologia da rede.

A reconfiguração ótima é uma importante ferramenta para aumentar a confiabilidade de um SDEE, especialmente quando a automação avançada e tecnologias de redes inteligentes (smartgrids) tornam-se mais importante e mais acessível às concessionarias de distribuição.

Os benefícios ao reduzir as perdas de potência ativa no sistema de distribuição:

\begin{itemize}
    \item Alívio do sistema de distribuição: com a redução das perdas de potência ativa, o sistema é aliviado, o que leva a uma maior vida útil dos equipamentos, uma maior capacidade de fornecimento e a um melhor perfil da magnitude de tensão no sistema;
    
    \item Adiamento de investimentos para a expansão do sistema de distribuição: a redução das perdas de potência tem como consequência a redução dos fluxos de potência nos condutores, e desta forma é adiada a necessidade de reforços na rede.
    
    \item Melhoria na qualidade de energia: a reconfiguração melhora o perfil da magnitude de tensão do sistema;
    
    \item Adiantamento da necessidade de ampliação da capacidade de transmissão: a rede de distribuição pode reduzir o carregamento de linhas de transmissão no horário de pico, aumentando efetivamente a capacidade de transmissão;
    
    \item Adiamento da ampliação da capacidade de geração: menos unidades de geração operando são necessárias no horário de pico;
    
    \item Redução do uso de combustíveis: ao reduzir as perdas, reduz-se a necessidade de geração de energia a partir de fontes não renováveis, o que leva a uma economia no uso de combustíveis fósseis;
    
    \item Benefícios ambientais: a redução no uso de combustíveis fósseis tem como consequência a redução de poluição;
    
    \item Redução na contratação de energia elétrica para grandes clientes: ao reduzir as perdas das redes dos grandes clientes, reduz-se o consumo de energia elétrica.
\end{itemize}

%Trabalho realizados anteriormente