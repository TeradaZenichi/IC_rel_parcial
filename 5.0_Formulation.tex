\section{Modelo matemático do problema}
\subsection{Modelagem do problema}


Com base nas deduções e hipóteses adotadas anteriormente, o problema de reconfiguração de uma rede elétrica radial pode ser representado utilizando um modelo de programação linear inteiro misto (PNLIM), mostrado a seguir:

\begin{equation*}
    \text{Min} = c^{lss}\sum_{ij\in\Omega_{l}}R_{ij}I_{ij}^{sqr}
\end{equation*}

Sujeito a:

\begin{equation*}
    \sum_{ji\in\Omega_{l}}P_{ji} - \sum_{ij\in\Omega_{l}}(P_{ij} + R_{ij}I_{ij}^{sqr})+ \sum_{ji\in\Omega_{ch}}P_{ji}^{ch} -\sum_{ij\in\Omega_{ch}}P_{ij}^{ch} + P_{i}^{S} = P_{i}^{D}\quad\forall i \in\Omega_{b}  
\end{equation*}
    
\begin{equation*}
    \sum_{ji\in\Omega_{l}}Q_{ji} - \sum_{ij\in\Omega_{l}}(Q_{ij} + X_{ij}I_{ij}^{sqr})+ \sum_{ji\in\Omega_{ch}}Q_{ji}^{ch} -\sum_{ij\in\Omega_{ch}}Q_{ij}^{ch} + Q_{i}^{S} = Q_{i}^{D}\quad\forall i \in\Omega_{b}
\end{equation*}

\begin{equation*}
    V_{i}^{sqr} - 2(R_{ij}P_{ij} + X_{ij}Q_{ij}) - Z_{ij}^{2}I_{ij}^{sqr} - V_{j}^{sqr} = 0\quad\forall ij \in \Omega_{l}
\end{equation*}

\begin{equation*}
    V_{j}^{sqr}I_{ij}^{sqr} = P_{ij}^{2}+Q_{ij}^{2}\quad\forall ij \in \Omega_{l}
\end{equation*}

\begin{equation*}
    -(\overline{V}^{2} - \underline{V}^{2})(1-w_{ij}) \leq V_{i}^{sqr} - V_{j}^{sqr} \leq (\overline{V}^{2} - \underline{V}^{2})(1-w_{ij})\qquad ij\in\Omega_{ch}        
\end{equation*}
    
\begin{equation*}
    -(\overline{V}\,\overline{I}_{ij}^{ch})w_{ij} \leq P_{ij}^{ch} \leq (\overline{V}\,\overline{I}_{ij}^{ch})w_{ij}\qquad ij\in\Omega_{ch}
\end{equation*}
    
    
\begin{equation*}
    -(\overline{V}\,\overline{I}_{ij}^{ch})w_{ij} \leq Q_{ij}^{ch} \leq (\overline{V}\,\overline{I}_{ij}^{ch})w_{ij}\qquad ij\in\Omega_{ch}   
\end{equation*}
    
\begin{equation*}
    |\Omega_{l}| + \sum_{ij\in\Omega_{ch}}w_{ij} = |\Omega_{b}| - 1
\end{equation*}

\begin{equation*}
    \underline{V}^{2} \leq V_{i}^{sqr} \leq \overline{V}^{2}\qquad i \in\Omega_{b}
\end{equation*}

\begin{equation*}
    0 \leq I_{ij}^{sqr} \leq \overline{I}_{ij}^{2} \qquad ij\in\Omega_{l} 
\end{equation*}

\begin{equation*}
    w_{ij}\quad\text{binário}\qquad\forall ij \in\Omega_{ch}
\end{equation*}

\subsection{Não linearidade}

A não linearidade desse problema está relacionada a restrição expressa na equação~\ref{eq:corrente_magnitude} a qual envolve o produto de duas variáveis quadráticas do sistema, bem como a presença de variáveis binárias ($w_{ij}$), caracterizando-se como um problema de programação não linear inteiro misto.
Dessa forma o problema não pode ser resolvido por \emph{solvers} lineares tais como CEPLEX e sim por programas capazes de resolve-lo como o Knitro e Bonmin, por exemplo.

