\section{Introdução}

Os sistemas de distribuição de energia (SDEE) são planejados como redes malhadas interconectadas.
Entretanto, operam como com uma topologia radial a fim da facilitar a coordenação da proteção e reduzir a corrente de curto circuito dos SDEE.
Para obter uma topologia radial existem chaves de interconexões em pontos estratégicos do sistema.
Desse modo, a topologia inicial pode ser modificada pela operação das chaves para transferir as demandas entre os diferentes alimentadores e, assim, é possível determinar uma nova topologia com outro ponto de operação, contudo deve continuar sendo uma topologia radial.

O problema de reconfiguração de sistemas de distribuição (RSD) consiste na abertura e/ou fechamento das chaves com o objetivo de melhorar um índice de desempenho. 
A reconfiguração ótima é uma importante ferramenta para aumentar a confiabilidade de um SDEE, especialmente quando a automação avançada e tecnologias de redes inteligentes (smartgrids) tornam-se mais importante e mais acessível às concessionarias de distribuição.





