\subsection{Balanço de potência}

\begin{figure}[H]
    \centering
    \includegraphics[scale = 1.4]{01_img/diagrama_potencia.png}
    \caption{Exemplo de um diagrama de balanço de potência entre nós de um sistema de distribuição de energia elétrica}
    \label{fig:balanco_pot}
\end{figure}

A figura~\ref{fig:balanco_pot} representa a forma expandida da figura~\ref{fig:SDR}, para melhor compreensão do problema de balanço de potência, onde $m$ e $n$ representam um número qualquer de nós ligados ao nó $i$. 
Para isso considere que nos parâmetros e variáveis que representam um ramo, o primeiro subíndice representa o nó de partida e o segundo subíndice o nó de chegada (exemplo: $P_{12}$ representa o fluxo de potência ativa que vai do nó 1 para o nó 2).

Seja $P_{i}^{D}$ e $Q_{i}^{D}$ potência ativa e reativa demandada no nó i respectivamente e $P_{i}^{S}$ e $Q_{i}^{S}$ potência ativa e reativa gerada no nó i respectivamente, têm-se que:

\begin{equation*}
    \sum_{ki\in\Omega_{l}}P_{ki} - \sum_{ij\in\Omega_{l}}(P_{ij} + R_{ij}I_{ij}^{2}) + P_{i}^{S} = P_{i}^{D}\quad\forall i \in\Omega_{b}
\end{equation*}

\begin{equation*}
    \sum_{ki\in\Omega_{l}}Q_{ki} - \sum_{ij\in\Omega_{l}}(Q_{ij} + X_{ij}I_{ij}^{2}) + Q_{i}^{S} = Q_{i}^{D}\quad\forall i \in\Omega_{b}
\end{equation*}

Dado que $\Omega_{b}$ é o conjunto de nós do sistema.
É possível mudar a variável \textit{k} pela variável \textit{j}, uma vez que ambas pertencem ao conjunto $\Omega_{l}$ e os somatórios envolvendo-as estão desconectadas, desse modo:

\begin{equation}
    \sum_{ji\in\Omega_{l}}P_{ji} - \sum_{ij\in\Omega_{l}}(P_{ij} + R_{ij}I_{ij}^{2}) + P_{i}^{S} = P_{i}^{D}\quad\forall i \in\Omega_{b}\label{eq:fluxo_pot_ativa}  
\end{equation}


\begin{equation}
    \sum_{ji\in\Omega_{l}}Q_{ji} - \sum_{ij\in\Omega_{l}}(Q_{ij} + X_{ij}I_{ij}^{2}) + Q_{i}^{S} = P_{i}^{D}\quad\forall i \in\Omega_{b}\label{eq:fluxo_pot_reativa}
\end{equation}

O sistema de equações não lineares em \ref{eq:fluxo_pot_ativa} e \ref{eq:fluxo_pot_reativa} representam a operação em regime permanente de uma rede elétrica radial e são frequentemente utilizados no método de varredura de fluxo de carga \cite{Shirmohammadi1988ANetworks} e \cite{Cespedes1990NewNetworks}