\section{Conclusão e próxima etapa}

Com esse projeto foi possível observar como a reconfiguração de uma rede de distribuição de energia elétrica que opera de forma radial pode reduzir de forma considerável as perdas de potência ativa. 
Além disso é possível observar como a reconfiguração do sistema de distribuição de energia elétrica manteve o nível de tensão em um intervalo determinado por norma em todos as barras do sistema.

Embora o o programa tenha encontrado uma solução ótima, é muito complicado achar a convergência desse tipo de problema para sistemas muito maiores do que esse. 
Como em centros urbanos a quantidade de nós é muito maior do que o problema visto é necessário utilizar um algoritmo diferente do \emph{solver} utilizado ou modelar o problema de forma a desaparecer com as não linearidades, uma vez que problemas lineares nunca divergem e são muito mais rápidos de serem calculados.

Por fim, para a próxima etapa do projeto será desenvolvido uma modelagem fundamentado na eliminação das não linearidades do problema de forma a transformar um problema PNLIM em um problema PLIM.


\subsection{Cronograma}

Os próximos passos para a segunda parte do projeto são:

\begin{itemize}
    \item Transformação do problema PNLIM em um problema PLIM
    
    \item Desenvolvimento de um modelo matemático de PLIM para resolver o problema de reconfiguração das redes de distribuição de energia elétrica.
    
    \item Realização dos estudos propostos em sistemas típicos de redes de distribuição e redes de distribuição reais. 
\end{itemize}