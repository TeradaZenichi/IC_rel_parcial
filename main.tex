\documentclass[a4paper, 12pt]{article}
%%%%%%%%%%%%%%%%%%%%%%%%%%%%%%%%%%%%%%%%%%%%%%%%%%%%%%%%%%%%%%%%%%%
    %Pacotes carregados%
%%%%%%%%%%%%%%%%%%%%%%%%%%%%%%%%%%%%%%%%%%%%%%%%%%%%%%%%%%%%%%%%%%%    
\usepackage[utf8]{inputenc}         %Permite acentuação
\usepackage[brazilian]{babel}       %Identifica erro gramatical e mudança de linguagem
\usepackage{amsmath}
\usepackage{graphicx}               %Importação de imagem
\usepackage{cite}                   %Pacote de citações
\usepackage{makeidx}                %Pacote para gerar sumário
\usepackage{enumerate}              %Pacote para utilizar o ambiente enumerate
\usepackage{indentfirst}            %Pacote que corrige o espaço dos parágrafos
\usepackage{setspace}               %Pacote de espaçamento entre linhas
\usepackage{enumitem}               %Pacote de ambiente itemize
\usepackage{hyperref}               %Pacote de gerenciamento de hiperlinks
\usepackage{url}                    %Pacote de gerenciamento de urls
\usepackage{colortbl}               %Pacote que permite o uso de cores nas tabelas
\usepackage{booktabs}               %Pacote para estilizar a coluna de tabelas
\usepackage{siunitx}                %Utilização de unidades do SI
\usepackage{placeins}               %Permite usar FloatBarrier pra delimitar até onde uma figura pode aparecer
%\usepackage[alf]{abntex2cite}      %Referências em ABNT... BUG
\usepackage[left=3.00cm, right=2.00cm, top=3.00cm, bottom=2.00cm]{geometry}                                                     %Margem padrão ABNT

\let\Oldsection\section
\renewcommand{\section}{\FloatBarrier\Oldsection} %Figuras de uma seção não podem aparecer em outra seção

\let\Oldsubsection\subsection
\renewcommand{\subsection}{\FloatBarrier\Oldsubsection} %Figuras de uma subseção não podem aparecer em outra subseção

\let\Oldsubsubsection\subsubsection
\renewcommand{\subsubsection}{\FloatBarrier\Oldsubsubsection} %Figuras de uma subsubseção não podem aparecer em outra subsubseção

\hypersetup{colorlinks = true, linkcolor=black, urlcolor = blue, citecolor = black}
\begin{document}
    \onehalfspacing                 %Para um espaçamento de 1,5 padrão ABNT
    \begin{titlepage}
\centering
    
{\large\textbf{Universidade Estadual de Campinas}
\par}
{\large\textbf{Faculdade de Engenharia Elétrica e de Computação}
\par}

\vspace{3.0cm}



%{\Large 
%Lucas Guimarães Braga \hfill RA: 182543\\
%Lucas Zenichi Terada \hfill RA: 182775\\
%Nícolas F. R. A. Prado \hfill RA: 185142\\
%Thiago H. C. da Cruz \hfill RA: 187576\\
%\par}

\vfill


\vspace{3cm}

{\large
Campinas, São Paulo, Brasil\\
\today \par}
\end{titlepage}
    \tableofcontents
    \section{Introdução}





    \section{Metodologia}

\subsection{Formulação do Problema de reconfiguração de Rede Elétrica}

%Inserir figuras de nós


    \section{Conclusão}
    \section{Resultados}

    %\nocite{*}                      %Adiciona referências à bibliografia sem terem que ser citadas
    
    \bibliographystyle{ieeetr}       %Estilo de apresentação da bibliografia
    \bibliography{6_ref.bib}            %Insere referencias ao texto
    
\end{document}
